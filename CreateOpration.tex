
\sloppy
{\fontsize{14pt}{14pt}\selectfont CREAT OPRATION \\} \par
\vspace{14pt}
\noindent 
{\fontsize{14pt}{14pt}\selectfont Dasar Git \\} \par
\noindent 
{\fontsize{14pt}{14pt}\selectfont Jadi, sebenarnya apa yang dimaksud dengan Git? Ini adalah bagian penting untuk dipahami, karena jika anda memahami apa itu Git dan cara kerjanya, maka dapat dipastikan anda dapat menggunakan Git secara efektif dengan mudah. Selama mempelajari Git, cobalah untuk melupakan VCS lain yang mungkin telah anda kenal sebelumnya, misalnya Subversion dan Perforce. Git sangat berbeda dengan sistem-sistem tersebut dalam hal menyimpan dan memperlakukan informasi yang digunakan, walaupun antar-muka penggunanya hampir mirip. Dengan memahami perbedaan tersebut diharapkan dapat membantu anda menghindari kebingungan saat menggunakan Git. \\} \par

 \par
\noindent 
{\fontsize{14pt}{14pt}\selectfont Salah satu perbedaan yang mencolok antar Git dengan VCS lainnya (Subversion dan kawan-kawan) adalah dalam cara Git memperlakukan datanya. Secara konseptual, kebanyakan sistem lain menyimpan informasi sebagai sebuah daftar perubahan berkas. Sistem seperti ini (CVS, Subversion, Bazaar, dan yang lainnya) memperlakukan informasi yang disimpannya sebagai sekumpulan berkas dan perubahan yang terjadi pada berkas-berkas tersebut, \\} \par
\vspace{14pt}
\noindent 
{\fontsize{14pt}{14pt}\selectfont Git tidak bekerja seperti ini. Melainkan, Git memperlakukan datanya sebagai sebuah kumpulan snapshot dari sebuah miniatur sistem berkas. Setiap kali anda melakukan commit, atau melakukan perubahan pada proyek Git anda, pada dasarnya Git merekam gambaran keadaan berkas-berkas anda pada saat itu dan menyimpan referensi untuk gambaran tersebut. Agar efisien, jika berkas tidak mengalami perubahan, Git tidak akan menyimpan berkas tersebut melainkan hanya pada file yang sama yang sebelumnya telah disimpan. \\} \par
\vspace{14pt}
\noindent 
{\fontsize{14pt}{14pt}\selectfont erbedaan penting antara Git dengan hampir semua VCS lain. Hal ini membuat Git mempertimbangkan kembali hampir setiap aspek dari version control yang oleh kebanyakan sistem lainnya disalin dari generasi sebelumnya. Ini membuat Git lebih seperti sebuah miniatur sistem berkas dengan beberapa tool yang luar biasa ampuh yang dibangun di atasnya, ketimbang sekadar sebuah VCS. $  $ \\} \par

 \par
\noindent 
{\fontsize{14pt}{14pt}\selectfont Kebanyakan operasi pada Git hanya membutuhkan berkas-berkas dan resource lokal – tidak ada informasi yang dibutuhkan dari komputer lain pada jaringan anda. Jika Anda terbiasa dengan VCS terpusat dimana kebanyakan operasi memiliki overhead latensi jaringan, aspek Git satu ini akan membuat anda berpikir bahwa para dewa kecepatan telah memberkati Git dengan kekuatan. Karena anda memiliki seluruh sejarah dari proyek di lokal disk anda, dengan kebanyakan operasi yang tampak hampir seketika. \\} \par
\noindent 
{\fontsize{14pt}{14pt}\selectfont Sebagai contoh, untuk melihat history dari proyek, Git tidak membutuhkan data histori dari server untuk kemudian menampilkannya untuk anda, namun secara sedarhana Git membaca historinya langsung dari basisdata lokal proyek tersebut. Ini berarti anda melihat histori proyek hampir secara instant. Jika anda ingin membandingkan perubahan pada sebuah berkas antara versi saat ini dengan versi sebulan yang lalu, Git dapat mencari berkas yang sama pada sebulan yang lalu dan melakukan pembandingan perubahan secara lokal, bukan dengan cara meminta remote server melakukannya atau meminta server mengirimkan berkas versi yang lebih lama kemudian membandingkannya secara lokal. \\} \par
\noindent 
{\fontsize{14pt}{14pt}\selectfont Hal ini berarti bahwa sangat sedikit yang tidak bisa anda kerjakan jika anda sedang offline atau berada diluar VPN. Jika anda sedang berada dalam pesawat terbang atau sebuah kereta dan ingin melakukan pekerjaan kecil, anda dapat melakukan commit sampai anda memperoleh koneksi internet hingga anda dapat menguploadnya. Jika anda pulang ke rumah dan VPN client anda tidak bekerja dengan benar, anda tetap dapat bekerja. Pada kebanyakan sistem lainnya, melakukan hal ini cukup sulit atau bahkan tidak mungkin sama sekali. Pada Perforce misalnya, anda tidak dapat berbuat banyak ketika anda tidak terhubung dengan server; pada Subversion dan CVS, anda dapat mengubah berkas, tapi anda tidak dapat melakukan commit pada basisdata anda (karena anda tidak terhubung dengan basisdata). Hal ini mungkin saja bukanlah masalah yang besar, namun anda akan terkejut dengan perbedaan besar yang disebabkannya. \\} \par

 \par
\noindent 
{\fontsize{14pt}{14pt}\selectfont Segala sesuatu pada Git akan melalui proses checksum terlebih dahulu sebelum disimpan yang kemudian direferensikan oleh hasil checksum tersebut. Hal ini berarti tidak mungkin melakukan perubahan terhadap berkas manapun tanpa diketahui oleh Git. Fungsionalitas ini dimiliki oleh Git pada level terendahnya dan ini merupakan bagian tak terpisahkan dari filosofi Git. Anda tidak akan kehilangan informasi atau mendapatkan file yang cacat tanpa diketahui oleh Git. \\} \par
\vspace{14pt}
\vspace{14pt}
\vspace{12pt}
\vspace{12pt}
\noindent 
{\fontsize{14pt}{14pt}\selectfont \underline{Secara Umum Git Hanya Menambahkan Data} \\} \par
\noindent 
{\fontsize{14pt}{14pt}\selectfont Ketika anda melakukan operasi pada Git, kebanyakan dari operasi tersebut hanya menambahkan data pada basisdata Git. It is very difficult to get the system to do anything that is not undoable or to make it erase data in any way. Seperti pada berbagai VCS, anda dapat kehilangan atau mengacaukan perubahan yang belum di-commit; namun jika anda melakukan commit pada Git, akan sangat sulit kehilanngannya, terutama jika anda secara teratur melakukan push basisdata anda pada repositori lain. \\} \par
\noindent 
{\fontsize{14pt}{14pt}\selectfont Hal ini menjadikan Git menyenangkan karena kita dapat berexperimen tanpa kehawatiran untuk mengacaukan proyek. Untuk lebih jelas dan dalam lagi tentang bagaimana Git menyimpan datanya dan bagaimana anda dapat mengembalikan yang hilang, lihat "Under the Covers" pada Bab 9. \\} \par

 \par
\noindent 
{\fontsize{14pt}{14pt}\selectfont Sekarang perhatikan. Ini adalah hal utama yang harus diingat tentang Git jika anda ingin proses belajar anda berjalan lancar. Git memiliki 3 keadaan utama dimana berkas anda dapat berada: committed, modified dan staged. Committed berarti data telah tersimpan secara aman pada basisdata lokal. Modified berarti anda telah melakukan perubahan pada berkas namun anda belum melakukan commit pada basisdata. Staged berarti anda telah menandai berkas yang telah diubah pada versi yang sedang berlangsung untuk kemudian dilakukan commit. \\} \par
\vspace{14pt}
\vspace{14pt}
\noindent 
{\fontsize{14pt}{14pt}\selectfont Direktori Git adalah dimana Git menyimpan metadata dan database objek untuk projek anda. Ini adalah bahagian terpenting dari Git, dan inilah yang disalin ketika anda melakukan kloning sebuah repository dari komputer lain. \\} \par
\noindent 
{\fontsize{14pt}{14pt}\selectfont Direktori kerja adalah sebuah checkout tunggal dari satu versi dari projek. Berkas-berkas ini kemudian ditarik keluar dari basisdata yang terkompresi dalam direktori Git dan disimpan pada disk untuk anda gunakan atau modifikasi. \\} \par
\noindent 
{\fontsize{14pt}{14pt}\selectfont Staging area adalah sebuah berkas sederhana, umumnya berada dalam direktori Git anda, yang menyimpan informasi mengenai apa yang menjadi commit selanjutnya. Ini terkadang disebut sebagai index, tetapi semakin menjadi standard untuk menyebutnya sebagai staging area. \\} \par
\noindent 
{\fontsize{14pt}{14pt}\selectfont Alur kerja dasar Git adalah seperti ini: \\} \par
\noindent 
{\fontsize{14pt}{14pt}\selectfont Anda mengubah berkas dalam direktori kerja anda. \\} \par
\noindent 
{\fontsize{14pt}{14pt}\selectfont Anda membawa berkas ke stage, menambahkan snapshotnya ke staging area. \\} \par
\noindent 
{\fontsize{14pt}{14pt}\selectfont Anda melakukan commit, yang mengambil berkas seperti yang ada di staging area dan menyimpan snapshotnya secara permanen ke direktori Git anda. \\} \par
\noindent 
{\fontsize{14pt}{14pt}\selectfont Jika sebuah versi tertentu dari sebuah berkas telah ada di direktori git, ia dianggap 'committed'. Jika berkas diubah (modified) tetapi sudah ditambahkan ke staging area, maka itu adalah 'staged'. Dan jika berkas telah diubah sejak terakhir dilakukan checked out tetapi belum ditambahkan ke staging area maka itu adalah 'modified' \\} \par
\vspace{14pt}
\vspace{14pt}
\noindent 
{\fontsize{14pt}{14pt}\selectfont Mari memulai menggunakan Git. Pertama, tentu saja anda harus menginstallnya terlebih dahulu. Anda dapat melakukan melalui berbagai cara; dua cara paling poluler adalah menginstallnya dari kode sumbernya atau menginstalkan paket yang telah disediakan untuk platform anda. \\} \par

 \par
\noindent 
{\fontsize{14pt}{14pt}\selectfont Jika anda dapat melakukannya, akan sangat berguna untuk dapat menginstallnya dari kode sumber, karena anda akan mendapatkan versi terbaru dari Git. Setiap versi dari Git cenderung akan menampilkan kemajuan pada sisi antarmuka pengguna, jadi menggunakan versi terbaru seringkali menjadi jalan terbaik jika anda terbiasa melakukan kompilasi perangkat lunak dari kode sumbernya. Dan juga menjadi masalah bahwa banyak distribusi Linux yang menyertakan versi Git yang sangat lama; kecuali anda mempergunakan distribusi Linux paling up-to-date atau menggunakan backport, menginstall dari kode sumbernya mungkin menjadi solusi terbaik. \\} \par
\vspace{14pt}
\noindent 
{\fontsize{14pt}{14pt}\selectfont Setup Git Untuk Pertama Kalinya \\} \par
\noindent 
{\fontsize{14pt}{14pt}\selectfont Sekarang anda telah memiliki Git pada sistem anda, berikutnya anda akan harus melakukan beberapa penyesuai pada lingkungan Git anda. Anda hanya perlu melakukan hal ini sekali saja; pada saat memperbaharui versi Git anda, penyesuai tidak perlu dilakukan lagi. Anda pun dapat mengubah penyesuaian tersebut setiap saat. \\} \par
\noindent 
{\fontsize{14pt}{14pt}\selectfont Pada Git terdapat sebuah perkakas yang disebut dengan git config yang memungkinkan anda untuk memperoleh informasi dan menetapkan variable konfigurasi yang mengontrol segala aspek bagaimana Git beroperasi dan berperilaku. Variable-variable ini dapat disimpan pada tiga tempat berbeda: gitconfig $  $file: Menyimpan berbagai nilai-nilai variable untuk setiap pengguna pada sistem dan semua repositori milik para pengguna tersebut. Jika anda memberikan opsi $  $--system $  $pada $  $git config, maka Git akan membaca dan menulis file konfigurasi ini secara spesifik. gitconfig $  $file: Spesifik hanya untuk pengguna yang bersangkutan. Anda dapat membuat Git membaca dan menulis pada berkas ini secara spesifik dengan memberikan opsi $  $--global. \\} \par
\noindent 
{\fontsize{14pt}{14pt}\selectfont config file pada direktori git (yaitu, $  $.git config) atau reposotori manapun yang sedang anda gunakan: Spesifik hanya pada repositori itu saja. Setiap nilai pada setiap tingkat akan selalu menimpa nilai yang telah ditetapkan pada level sebelumnya, jadi nilai yang telah di-set pada $  $.git configakan menimpa nilai yang telah di-set pada $  $gitconfig. \\} \par
\vspace{14pt}
\noindent 
{\fontsize{14pt}{14pt}\selectfont Apakah Branch Itu \\} \par
\noindent 
{\fontsize{14pt}{14pt}\selectfont Untuk benar-benar mengerti cara Git melakukan branching, kita perlu kembali ke belakang dan membahas bagaimana Git menyimpan datanya. Seperti yang mungkin anda ingat dari Bab 1, Git tidak menyimpan data sebagai serangkaian kumpulan perubahan atau delta, melainkan sebagai serangkaian snapshot. \\} \par
\noindent 
{\fontsize{14pt}{14pt}\selectfont Ketika anda melakukan commit dalam Git, Git menyimpan sebuah object commit yang berisi pointer ke snapshot dari konten yang anda staged, metadata pembuat (author) dan pesan (message), dan nol atau lebih pointer ke commit yang merupakan parent (induk) langsung dari commit ini: nol jika ini commit yang pertama, satu jika ini commit yang normal, dan beberapa jika ini commit yang dihasilkan dari gabungan antara dua atau lebih branch. \\} \par
\noindent 
{\fontsize{14pt}{14pt}\selectfont Untuk memvisualisasikan ini, mari kita asumsikan anda memiliki direktori yang berisi tiga buah berkas, dan anda menambahkan mereka ke stage dan melakukan commit. Proses staging berkas melakukan checksum (dengan hash SHA-1 yang telah kita sebutkan di Bab 1), menyimpan versi berkas tersebut dalam repositori Git (Git merujuknya sebagai 'blobs') \\} \par
\noindent 
{\fontsize{14pt}{14pt}\selectfont Apa itu version control, dan kenapa anda harus peduli? Version control adalah sebuah sistem yang mencatat setiap perubahan terhadap sebuah berkas atau kumpulan berkas sehingga pada suatu saat anda dapat kembali kepada salah satu versi dari berkas tersebut. Sebagai contoh dalam buku ini anda akan menggunakan kode sumber perangkat lunak sebagai berkas yang akan dilakukan version controlling, meskipun pada kenyataannya anda dapat melakukan ini pada hampir semua tipe berkas di komputer. \\} \par
\noindent 
{\fontsize{14pt}{14pt}\selectfont Jika anda adalah seorang desainer grafis atau desainer web dan anda ingin menyimpan setiap versi dari gambar atau layout yang anda buat (kemungkinan besar anda pasti ingin melakukannya), maka Version Control System (VCS) merupakan sebuah solusi bijak untuk digunakan. Sistem ini memungkinkan anda untuk mengembalikan berkas anda pada kondisi/keadaan sebelumnya, mengembalikan seluruh proyek pada keadaan sebelumnya, membandingkan perubahan setiap saat, melihat siapa yang terakhir melakukan perubahan terbaru pada suatu objek sehingga berpotensi menimbulkan masalah, siapa yang menerbitkan isu, dan lainnya. Dengan menggunakan VCS dapat berarti jika anda telah mengacaukan atau kehilangan berkas, anda dapat dengan mudah mengembalikannya. Ditambah lagi, anda mendapatkan semua ini dengan overhead yang sangat sedikit. \\} \par

 \par
\noindent 
{\fontsize{14pt}{14pt}\selectfont Kebanyakan orang melakukan pengontrolan versi dengan cara menyalin berkas-berkas pada direktori lain (mungkin dengan memberikan penanggalan pada direktori tersebut, jika mereka rajin). Metode seperti ini sangat umum karena sangat sederhana, namun cenderung rawan terhadap kesalahan. Anda akan sangat mudah lupa dimana direktori anda sedang berada, selain itu dapat pula terjadi ketidak sengajaan penulisan pada berkas yang salah atau menyalin pada berkas yang bukan anda maksudkan. \\} \par
\noindent 
{\fontsize{14pt}{14pt}\selectfont Untuk mengatasi permasalahan ini, para programmer mengembangkan berbagai VCS lokal yang memiliki sebuah basis data sederhana untuk menyimpan semua perubahan pada berkas yang berada dalam cakupan revision control \\} \par
\vspace{14pt}
\noindent 
{\fontsize{14pt}{14pt}\selectfont Mengambil Repositori Git \\} \par
\noindent 
{\fontsize{14pt}{14pt}\selectfont Anda dapat mengambil sebuah proyek Git melalui 2 pendekatan utama. Cara pertama adalah dengan mengambil proyek atau direktori tersedia untuk dimasukkan ke dalam Git. Cara kedua adalah dengan melakukan kloning/duplikasi dari repositori Git yang sudah ada dari server. \\} \par

 \par
\noindent 
{\fontsize{14pt}{14pt}\selectfont Jika Anda mulai memantau proyek yang sudah ada menggunakan Git, Anda perlu masuk ke direktori dari proyek tersebut dan mengetikkan \\} \par
\noindent 
{\fontsize{14pt}{14pt}\selectfont  $  \$  $ git init \\} \par
\noindent 
{\fontsize{14pt}{14pt}\selectfont Git akan membuat sebuah subdirektori baru bernama .git yang akan berisi semua berkas penting dari repositori Anda, yaitu kerangka repositori dari Git. Pada titik ini, belum ada apapun dari proyek Anda yang dipantau. (Lihat Bab 9 untuk informasi lebih lanjut mengenai berkas apa saja yang terdapat di dalam direktori $  $.git $  $yang baru saja kita buat.) \\} \par
\noindent 
{\fontsize{14pt}{14pt}\selectfont Jika Anda ingin mulai mengendalikan versi dari berkas tersedia (bukan direktori kosong), Anda lebih baik mulai memantau berkas tersebut dengan melakukan commit awal. Caranya adalah dengan beberapa perintah $  $git add $  $untuk merumuskan berkas yang ingin anda pantau, diikuti dengan sebuah commit: \\} \par
\noindent 
{\fontsize{14pt}{14pt}\selectfont git add *.c \\} \par
\noindent 
{\fontsize{14pt}{14pt}\selectfont git add README \\} \par
\noindent 
{\fontsize{14pt}{14pt}\selectfont it commit –m 'versi awal proyek' \\} \par
\noindent 
{\fontsize{14pt}{14pt}\selectfont Kita akan membahas apa yang dilakukan perintah-perintah di atas sebentar lagi. Pada saat ini, Anda sudah memiliki sebuah repositori Git berisi file-file terpantau dan sebuah commit awal. \\} \par

 \par
\noindent 
{\fontsize{14pt}{14pt}\selectfont Jika Anda ingin membuat salinan dari repositori Git yang sudah tersedia — misalnya, dari sebuah proyek yang Anda ingin ikut berkontribusi di dalamnya — perintah yang Anda butuhkan adalah $  $git clone. Jika Anda sudah terbiasa dengan sistem VCS lainnya seperti Subversion, Anda akan tersadar bahwa perintahnya adalah clone dan bukan checkout. Ini adalah pembedaan yang penting — Git menerima salinan dari hampir semua data yang server miliki. Setiap versi dari setiap berkas yang tercatat dalam sejarah dari proyek tersebut akan ditarik ketika Anda menjalankan $  $git clone. Bahkan, ketika cakram di server Anda rusak, Anda masih dapat menggunakan hasil duplikasi di klien untuk mengembalikan server Anda ke keadaan tepat pada saat duplikasi dibuat (Anda mungkin kehilangan beberapa hooks atau sejenisnya yang sebelumnya telah ditata di sisi server, namun semua versi data sudah kembali seperti sediakala-lihat Bab 4 untuk lebih detil). \\} \par
\noindent 
{\fontsize{14pt}{14pt}\selectfont Anda menduplikasi sebuah repositori menggunakan perintah $  $git clone [url]. Sebagai contoh, jika Anda ingin menduplikasi pustaka Git Ruby yang disebut Grit, Anda dapat melakukannya sebagai berikut: \\} \par
\noindent 
{\fontsize{14pt}{14pt}\selectfont git clone git://github.com/schacon/grit.git \\} \par
\noindent 
{\fontsize{14pt}{14pt}\selectfont Perintah ini akan membuat sebuah direktori yang dinamakan "grit", menata awal sebuah direktori $  $.gitdi dalamnya, menarik semua data dari repositori, dan $  $checkout $  $versi mutakhir dari salinan kerja. Jika Anda masuk ke dalam direktori $  $grit $  $tersebut, Anda akan melihat berkas-berkas proyek sudah ada di sana, siap untuk digunakan. Jika Anda ingin membuat duplikasi dari repositori tersebut ke direktori yang tidak dinamakan grit. \\} \par
\noindent 
{\fontsize{14pt}{14pt}\selectfont Perintah ini bekerja seperti perintah sebelumnya, namun direktori tujuannya akan diberi nama mygrit. \\} \par
\noindent 
{\fontsize{14pt}{14pt}\selectfont Git memiliki beberapa protokol transfer yang berbeda yang dapat digunakan. Pada contoh sebelumnya, kita menggunakan protokol $  $ yang akan menggunakan SSH sebagai protokol transfer. Bab 4 akan memperkenalkan Anda kepada semua opsi yang tersedia yang dapat ditata di sisi server untuk mengakses repositori Git Anda dan keuntungan dan kelebihan dari masing-masing protokol. \\} \par
\vspace{14pt}
\noindent 
{\fontsize{14pt}{14pt}\selectfont Jika Anda hanya sempat membaca satu bab untuk dapat bekerja dengan Git, bab inilah yang tepat. Bab ini menjelaskan setiap perintah dasar yang Anda butuhkan untuk menyelesaikan sebagian besar permasalahan yang akan Anda hadapi dalam penggunaan Git. Pada akhir bab, Anda akan dapat mengkonfigurasi dan memulai sebuah repositori, memulai dan mengakhiri pemantauan berkas, dan melakukan staging dan committing perubahannya. Kami juga akan menunjukkan kepada Anda cara menata Git untuk mengabaikan berkas-berkas ataupun pola berkas tertentu, cara untuk membatalkan kesalahan secara cepat dan mudah, cara untuk melihat sejarah perubahan dari proyek dan melihat perubahan-perubahan yang telah terjadi diantara commit, dan cara untuk mendorong dan menarik perubahan dari repositori lain. \\} \par
\vspace{14pt}
\noindent 
{\fontsize{14pt}{14pt}\selectfont Merekam Perubahan ke dalam Repositori \\} \par
\noindent 
{\fontsize{14pt}{14pt}\selectfont Anda sudah memiliki repositori Git yang bonafide dan sebuah salinan kerja dari semua berkas untuk proyek tersebut. Anda harus membuat beberapa perubahan dan commit perubahan tersebut ke dalam repositori setiap saat proyek mencapai sebuah keadaan yang ingin Anda rekam. \\} \par
\noindent 
{\fontsize{14pt}{14pt}\selectfont Ingat bahwa setiap berkas di dalam direktori kerja Anda dapat berada di 2 keadaan: terpantau atau tak-terpantau. Berkas terpantau adalah berkas yang sebelumnya berada di snapshot terakhir; mereka dapat berada dalam kondisi belum terubah, terubah, ataupun staged (berada di area stage). Berkas tak-terpantau adalah kebalikannya - merupakan berkas-berkas di dalam direktori kerja yang tidak berada di dalam snapshot terakhir dan juga tidak berada di area staging. Ketika Anda pertama kali menduplikasi sebuah repositori, semua berkas Anda akan terpantau dan belum terubah karena Anda baru saja melakukan checkout dan belum mengubah apapun. \\} \par

 \par
\noindent 
{\fontsize{14pt}{14pt}\selectfont Alat utama yang Anda gunakan untuk menentukan berkas-berkas mana yang berada dalam keadaan tertentu adalah melalui perintah $  $git status. Jika Anda menggunakan alat ini langsung setelah sebuah $  $clone, Anda akan melihat serupa seperti di bawah ini: \\} \par
\noindent 
{\fontsize{14pt}{14pt}\selectfont git status \\} \par
\noindent 
{\fontsize{14pt}{14pt}\selectfont On branch master \\} \par
\noindent 
{\fontsize{14pt}{14pt}\selectfont nothing to commit, working directory clean \\} \par
\noindent 
{\fontsize{14pt}{14pt}\selectfont Ini berarti Anda memiliki direktori kerja yang bersih-dengan kata lain, tidak ada berkas terpantau yang terubah. Git juga tidak melihat berkas-berkas yang tak terpantau, karena pasti akan dilaporkan oleh alat ini. Juga, perintah ini memberitahu Anda tentang cabang tempat Anda berada. Pada saat ini, cabang akan selalu berada di master, karena sudah menjadi default-nya; Anda tidak perlu khawatir tentang cabang dulu. Bab berikutnya akan membahas tentang percabangan dan referensi secara lebih detil. \\} \par

 \par
\noindent 
{\fontsize{14pt}{14pt}\selectfont Untuk mulai memantau berkas baru, Anda menggunakan perintah $  $git add. Untuk mulai memantau berkas README tadi, Anda menjalankannya seperti berikut: \\} \par
\noindent 
{\fontsize{14pt}{14pt}\selectfont git add README \\} \par
\noindent 
{\fontsize{14pt}{14pt}\selectfont Jika Anda menjalankan perintah $  $status $  $lagi, Anda akan melihat bahwa berkas README Anda sekarang sudah terpantau dan sudah masuk ke dalam area stage: \\} \par
\noindent 
{\fontsize{14pt}{14pt}\selectfont git status \\} \par
\noindent 
{\fontsize{14pt}{14pt}\selectfont On branch master \\} \par
\noindent 
{\fontsize{14pt}{14pt}\selectfont Changes to be committed: \\} \par
\noindent 
{\fontsize{14pt}{14pt}\selectfont use git reset HEAD <file>... to unstage \\} \par
\noindent 
{\fontsize{14pt}{14pt}\selectfont new~file:~  README \\} \par
\noindent 
{\fontsize{14pt}{14pt}\selectfont Anda dapat mengatakan bahwa berkas tersebut berada di dalam area stage karena tertulis di bawah judul Changes to be committed. Jika Anda melakukan commit pada saat ini, versi berkas pada saat Anda menjalankan $  $git add $  $inilah yang akan dimasukkan ke dalam sejarah snapshot. Anda mungkin ingat bahwa ketika Anda menjalankan $  $git init $  $sebelumnya, Anda melanjutkannya dengan $  $git add (nama berkas) $  $yang akan mulai dipantau di direktori Anda. Perintah $  $git add $  $ini mengambil alamat dari berkas ataupun direktori; jika sebuah direktori, perintah tersebut akan menambahkan seluruh berkas yang berada di dalam direktori secara rekursif. \\} \par
\vspace{14pt}
\vspace{14pt}
\noindent 
{\fontsize{14pt}{14pt}\selectfont Bekerja Berjarak \\} \par
\noindent 
{\fontsize{14pt}{14pt}\selectfont Untuk dapat berkolaborasi untuk proyek Git apapun, Anda perlu mengetahui bagaimana Anda dapat mengatur repositori berjarak dari jarak jauh. Repositori berjarak adalah sekumpulan versi dari proyek Anda yang disiarkan di Internet atau di jaringan. Anda dapat memiliki beberapa repositori berjarak, masing-masing bisanya dengan akses terbatas untuk membaca saja ataupun baca/tulis. Berkolaborasi dengan pihak lain menuntut kemampuan untuk mengatur repositori berjarak ini dan menarik dan mendorong data ke dan dari repositori berjarak tersebut ketika Anda butuh untuk membagi hasil kerja Anda. \\} \par
\vspace{14pt}
\noindent 
{\fontsize{14pt}{14pt}\selectfont Apakah Branch Itu \\} \par
\noindent 
{\fontsize{14pt}{14pt}\selectfont Untuk benar-benar mengerti cara Git melakukan branching, kita perlu kembali ke belakang dan membahas bagaimana Git menyimpan datanya. Seperti yang mungkin anda ingat dari Bab 1, Git tidak menyimpan data sebagai serangkaian kumpulan perubahan atau delta, melainkan sebagai serangkaian snapshot. \\} \par
\noindent 
{\fontsize{14pt}{14pt}\selectfont Ketika anda melakukan commit dalam Git, Git menyimpan sebuah object commit yang berisi pointer ke snapshot dari konten yang anda staged, metadata pembuat (author) dan pesan (message), dan nol atau lebih pointer ke commit yang merupakan parent (induk) langsung dari commit ini: nol jika ini commit yang pertama, satu jika ini commit yang normal, dan beberapa jika ini commit yang dihasilkan dari gabungan antara dua atau lebih branch. \\} \par
\vspace{14pt}
\noindent 
{\fontsize{14pt}{14pt}\selectfont Hampir setiap VCS memiliki sejumlah dukungan atas branching (percabangan). Branching adalah membuat cabang dari repositori utama dan melanjutkan melakukan pekerjaan pada cabang yang baru tersebut tanpa perlu khawatir mengacaukan yang utama. Dalam banyak VCS, branching adalah proses yang agak mahal, karena seringkali mengharuskan anda untuk membuat salinan baru dari direktori kode sumber, dimana dapat memakan waktu lama untuk proyek-proyek yang besar. \\} \par
\noindent 
{\fontsize{14pt}{14pt}\selectfont Beberapa orang menyebut model branching dalam Git sebagai "killer feature," hal inilah yang membuat Git berbeda di komunitas VCS. Mengapa begitu istimewa? Cara Git membuat cabang sangatlah ringan, membuat operasi branching hampir seketika dan berpindah bolak-balik antara cabang umumnya sama cepatnya. Tidak seperti VCS lainnya, Git mendorong alur kerja dimana kita sering membuat cabang dan kemudian menggabungkannya, bahkan dapat beberapa kali dalam sehari. Memahami dan menguasai fitur ini memberi anda perangkat yang ampuh, unik, dan benar-benar dapat mengubah cara anda melakukan pengembangan (develop). \\} \par
\vspace{14pt}
