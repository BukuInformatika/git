\section{Perintah Dasar Bash}
\subsection{fungsi perintah pada Bash}

\begin{enumerate}
\item ls
	fungsi perintah \textbf{ls} pada bash ini yaitu untuk melihat isi dari suatu direktori. 
Contoh lihat pada gambar \ref{ls}
\begin{figure}[!htbp]
\centerline{\includegraphics[width=.75\textwidth]{Figures/ls.jpg}}
\caption{Perintah dasar ls pada git bash}
\label{ls}
\end{figure}

\item cd 
	fungsi perintah \textbf{cd} \textit{(change directory)} pada bash ini yaitu untuk berpindah ke direktori yang dituju.
Contoh lihat pada gambar \ref{cd}
\begin{figure}[!htbp]
\centerline{\includegraphics[width=.75\textwidth]{Figures/cd.jpg}}
\caption{Perintah dasar cd pada git bash}
\label{cd}
\end{figure}

\item pwd 
	fungsi perintah \textbf{pwd} pada bash ini yaitu untuk mengetahui path direktori yang sedang aktif.
Contoh lihat pada gambar \ref{pwd}
\begin{figure}[!htbp]
\centerline{\includegraphics[width=.75\textwidth]{Figures/pwd.jpg}}
\caption{Perintah dasar pwd pada git bash}
\label{pwd}
\end{figure}

\item mv 
	fungsi perintah \textbf{mv} pada bash ini yaitu untuk mengubah nama file.
Contoh lihat pada gambar \ref{mv}
\begin{figure}[!htbp]
\centerline{\includegraphics[width=.75\textwidth]{Figures/mv.jpg}}
\caption{Perintah dasar mv pada git bash}
\label{mv}
\end{figure}

\item cp
	fungsi perintah \textbf{cp} pada bash ini yaitu untuk men\textit{copy file}.
Contoh lihat pada gambar \ref{cp}
\begin{figure}[!htbp]
\centerline{\includegraphics[width=.75\textwidth]{Figures/cp.jpg}}
\caption{Perintah dasar cp pada git bash}
\label{cp}
\end{figure}

\end{enumerate}

 